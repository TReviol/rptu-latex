\documentclass[]{beamer}
\usepackage[utf8]{inputenc}
\usepackage[T1]{fontenc}

\usetheme[dunkelgruen,
		  redhattext=true,
		  %displaynavigation,compress,
		  displayinstitute,  
		  ownlogo=false,         
		  displayframetotal]{rptu}
% Available options:
% See below for available options

\usepackage{lipsum}

%%%%%%%%%%%%%%%%%%%%%%%%%%%%%%%%%%%%%%%%%%%%%%%%%%%%%%%%%
%                  Main Configuration                   %
%%%%%%%%%%%%%%%%%%%%%%%%%%%%%%%%%%%%%%%%%%%%%%%%%%%%%%%%%

\title[RPTU \LaTeX\ Presentation]{A first RPTU \LaTeX\ Presentation}
\subtitle{with the new RPTU Kaiserslautern-Landau Corporate Design}
\date{Kaiserslautern, \today}
\author[Mustermann]{Dr. Max Mustermann}
\institute[Mathematik]{AG Statistik, FB Mathematik}

\renewcommand{\ownlogo}{}



%%%%%%%%%%%%%%%%%%%%%%%%%%%%%%%%%%%%%%%%%%%%%%%%%%%%%%%%%
%                    Begin Content                      %
%%%%%%%%%%%%%%%%%%%%%%%%%%%%%%%%%%%%%%%%%%%%%%%%%%%%%%%%%
\begin{document}
\begin{frame}
\titlepage
\end{frame}

\begin{frame}{Overview}
	\tableofcontents
\end{frame}

\section{Theme Options}
\rptusectionpage

\begin{frame}{Available options~I}
\framesubtitle{\texttt{\textbackslash usetheme[options]\{rptu\}}}
	\begin{tabular}{ll}
		\textbf{Option} & \textbf{Description}\\ \hline
		\texttt{frametotal=true|false} &Show|\textbf{Hide} total number of slides \\
		\texttt{displayframetotal} & Show total number of slide \\ 
		\texttt{hideframetotal} & Hide total number of slide \\  \hline
		\texttt{dunkelblau|hellblau} & Set color scheme of presentation\\
		\texttt{rot|orange} & (\textbf{dunkelblau} is default)\\
		\texttt{dunkelgruen|hellgruen} & Each line defines a colorscheme,\\
		\texttt{blaugrau|gruengrau} & you specify which is the main color\\ 
		\texttt{violett|pink} & and which is the secondary color\\ 
	\end{tabular}
	\vspace*{2ex}
	
	Defaults are the descriptions written in \textbf{bold}.
\end{frame}

\begin{frame}{Available options~II}
\framesubtitle{\texttt{\textbackslash usetheme[options]\{rptu\}}}
	\begin{tabular}{ll}
		\textbf{Option} & \textbf{Description}\\ \hline
		\texttt{navigation=true|false} &Show|\textbf{Hide} navigation in headline \\
		\texttt{displaynavigation} & Show navigation in headline \\ 
		\texttt{hidenavigation} & Hide navigation in headline \\  \hline
		\texttt{compress} & equivalent to beamer's compress \\ \hline
		\texttt{institute=true|false} &Show|\textbf{Hide} short institute in footline \\
		\texttt{displayinstitute} & Show short institute in footline \\ 
		\texttt{hideinstitute} & Hide short institute in footline \\ \hline
		\texttt{redhattext=true|false} & Use|\textbf{Do not use} the Red Hat Text Font \\ & (only tested with LuaLaTeX) \\ \hline
		\texttt{ownlogo=true|false} & Use|\textbf{Do not use} a user specified logo \\ & (instead of default)
	\end{tabular}
	\vspace*{2ex}
	
	Defaults are the descriptions written in \textbf{bold}.
\end{frame}

\begin{frame}{Options}{Some Remarks}
\begin{itemize}
	\item When using \texttt{navigation=true} you should use the option \texttt{compress} to obtain a single line for the navigation symbols. Otherwise each subsection has its own line. The complete navigation might not fit on the slide.
	\item If \texttt{institute=true} the \texttt{\textbackslash institute} is used to set the name of department/institute/affiliation both in the footline and on the title page:	
	
		\texttt{\textbackslash institute[name in footline]\{name on title page\}}
	 
	\item This presentation uses the options \texttt{dunkelblau},
		  \texttt{displayinstitute}, \texttt{displaynavigation},
		  \texttt{compress},
          \texttt{displayframetotal}
        \item Use \texttt{\textbackslash documentclass[aspectratio=169]\{beamer\}} for getting a 16:9 aspect ratio instead of 4:3
\end{itemize}
	\end{frame}


\begin{frame}{Using Red Hat Text}
	\begin{itemize}
		\item Make sure that the 'static' versions of the font are installed on your computer
		\item We use \texttt{RedHatText-Regular}, \texttt{RedHatText-Italic}, \texttt{RedHatText-SemiBold}, \texttt{RedHatText-SemiBoldItalic}. The corresponding individual \texttt{.ttf} files need to be available.
		\item Use LuaLaTex or Xetex, otherwise the default font is used instead of Red Hat Text
	\end{itemize}
\end{frame}

\section{Elements of a Frame}
\rptusectionpage
	
\subsection{Blocks}

\begin{frame}{Blocks}
	\begin{block}{Basic Block}
	\texttt{block}
	\end{block}
	\begin{alertblock}{Alert Block}
	\texttt{alertblock} (always in rptured/rptuorange)
	\end{alertblock}
\begin{exampleblock}{Example Block}
	\texttt{exampleblock}  (always in rptudunkelgruen/rptuhellgruen)
	\end{exampleblock}
\end{frame}

\begin{frame}
		\begin{definition}[Probability Space]
		\texttt{definition}
	\end{definition}
		\begin{Theorem}[Bayes' Theorem]
			Theorem
		\end{Theorem}
		\begin{Example}
			Here goes an example
		\end{Example}
		\begin{proof}
			And a proof.
		\end{proof}
\end{frame}

\begin{frame}{And more columns}
\begin{columns}
\begin{column}{0.3\textwidth}
\begin{block}{Type 1}	
Lorem ipsum dolor sit amet, consetetur sadipscing elitr, sed diam nonumy eirmod tempor invidunt ut labore et dolore magna aliquyam erat.
\end{block}
\end{column}
\begin{column}{0.3\textwidth}
\begin{block}{Type 2}
Sed diam voluptua. At vero eos et accusam et justo duo dolores et ea rebum. Stet clita kasd gubergren, no sea takimata sanctus est Lorem ipsum dolor sit amet. 
\end{block}
\end{column}
\begin{column}{0.3\textwidth}
\begin{block}{Type 3}
Sed diam voluptua. At vero eos et accusam et justo duo dolores et ea rebum. 
\end{block}
\end{column}
\end{columns}
\end{frame}

\begin{frame}{A RPTU Design Box}
\begin{rptu-twocolors}
\texttt{\textbackslash begin\{rptu-twocolors\} \\ text \\ \textbackslash end\{rptu-twocolors\}}
\end{rptu-twocolors}
	
	\begin{rptu-onecolor}
		\texttt{\textbackslash begin\{rptu-onecolor\} \\ text \\ \textbackslash end\{rptu-onecolor\}}
	\end{rptu-onecolor}
\end{frame}

\subsection{Footnotes}

\begin{frame}{Footnotes}
	Sed diam voluptua. At vero eos et accusam et justo duo dolores et ea rebum. Stet clita kasd gubergren, no sea takimata sanctus est Lorem ipsum dolor sit amet. Sed diam voluptua.\footnote{here goes a footnote} At vero eos et accusam et justo duo dolores et ea rebum. Stet clita kasd gubergren, no sea takimata sanctus est Lorem ipsum dolor sit amet.
\end{frame}

\begin{frame}{Footnotes}
\framesubtitle{What happens with mutiple footnotes?}
	Sed diam voluptua. At vero eos et accusam et justo duo dolores\footnote{some footnote} et ea rebum. Stet clita kasd gubergren, no sea takimata sanctus est Lorem ipsum dolor sit amet. Sed diam voluptua.\footnote{here goes a footnote} At vero eos et accusam et justo duo dolores et ea rebum. 
	
	Stet clita kasd gubergren, no sea takimata sanctus est Lorem ipsum dolor sit amet.\footnote{even a third footnote}
\end{frame}

\begin{frame}[plain]
	An example for a plain slide.
\end{frame}

\begin{frame}
A frame without a title but not plain	
\end{frame}

\subsection{Lists}

\begin{frame}{Itemize and Enumerate Environments}
\begin{columns}
\begin{column}{0.5\textwidth}
   \begin{itemize}
		\item one \begin{itemize}
		\item item one one
		\item item one two
		\end{itemize}
		\item two
	\end{itemize}
\end{column}
\begin{column}{0.5\textwidth}  %%<--- here
   \begin{enumerate}
\item one
\item two \begin{enumerate}
	\item enumerate with subitem
\end{enumerate}
\end{enumerate}
\end{column}
\end{columns}
\end{frame}

\section{Math}
\rptusectionpage

\begin{frame}{There Is No Largest Prime Number}
\framesubtitle{The proof uses \textit{reductio ad absurdum}.} 
\begin{theorem}
There is no largest prime number. \end{theorem} 

\pause
\begin{proof}
	\begin{enumerate} 
\item<2-| alert@2> Suppose $p$ were the largest prime number. 
\item<3-> Let $q$ be the product of the first $p$ numbers. 
\item<3-> Then $q+1$ is not divisible by any of them. 
\item<4-> But $q + 1$ is greater than $1$, thus divisible by some prime
number not in the first $p$ numbers.
\end{enumerate}
\end{proof}
\end{frame}


\section{Images and Figures}
\subsection*{and the current subsection}

\rptusectionpage

\begin{frame}{Include Images}
	\begin{columns}
	\begin{column}{0.3\textwidth}
		\begin{figure}
			\includegraphics[width=\linewidth]{example-image-a}
			\caption{Image A}
		\end{figure}
	\end{column}
	\begin{column}{0.3\textwidth}
		\begin{figure}
			\includegraphics[width=\linewidth]{example-image-b}
			\caption{Image B}
		\end{figure}
	\end{column}
	\begin{column}{0.3\textwidth}
		\begin{figure}
			\includegraphics[width=\linewidth]{example-image-b}
			\caption{Image C}
		\end{figure}
	\end{column}
	\end{columns}

\vspace{1em}
\end{frame}


\begin{frame}{Include Images}{Two rows of images}
	\begin{columns}
	\begin{column}{0.3\textwidth}
		\begin{figure}
			\includegraphics[width=\linewidth]{example-image-a}
			\includegraphics[width=\linewidth]{example-image-a}
		\end{figure}
	\end{column}
	\begin{column}{0.3\textwidth}
		\begin{figure}
			\includegraphics[width=\linewidth]{example-image-b}
			\includegraphics[width=\linewidth]{example-image-a}
		\end{figure}
	\end{column}
	\begin{column}{0.3\textwidth}
		\begin{figure}
			\includegraphics[width=\linewidth]{example-image-b}
			\includegraphics[width=\linewidth]{example-image-a}
		\end{figure}
	\end{column}
	\end{columns}
\end{frame}


\section{Other Macros}
\subsection*{and another subsection name}
\rptusectionpage

\begin{frame}
	\begin{itemize}
		\item Exclude certain slides from the navigation bar (if it is shown)
	\end{itemize}
\end{frame}

\navigationexclude
\begin{frame}
	This frame is not taken into account for the navigation.
	
	We used \texttt{\textbackslash navigationexclude} before this slide.
\end{frame}

\begin{frame}
	Neither is this one.
\end{frame}

\begin{frame}
	Or this.
\end{frame}

\navigationinclude
\begin{frame}
	But this frame is in the navigation again.
	
	We used \texttt{\textbackslash navigationinclude} before this slide.
\end{frame}

\begin{frame}
\vspace*{\baselineskip}
\lipsum[2-3][1-20]
 
And one more line.
	
And another one.
\end{frame}

\end{document}